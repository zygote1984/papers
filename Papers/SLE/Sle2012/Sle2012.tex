
% This is LLNCS.DOC the documentation file of
% the LaTeX2e class from Springer-Verlag
% for Lecture Notes in Computer Science, version 2.4
\documentclass{llncs}
\usepackage{llncsdoc}
\usepackage{color}
%
\begin{document}

\title{SLE2012}

\author{Kardelen Hatun \and Christoph Bockisch \and Mehmet Ak\c{s}it}
\institute{TRESE, University of Twente \\ 7500AE Enschede \\ The Netherlands \\
\url{http://www.utwente.nl/ewi/trese/}\\
\email{ \{hatunk,c.m.bockisch,aksit\}@ewi.utwente.nl}
}

\maketitle
\section{Introduction}

Component interact through their interfaces. When components are developed according to the same interface requirements their integration does not need additional programming. In component-based systems, there are cases where a certain functionality is provided by a third-party software which usually comes with an incompatible interface. In a case where one component needs to access data the other component does not share requires re-programming a part of the system. Such changes may cause a ripple effect and may cause other components to malfunction which are dependent on the changed one. So an important requirement of \emph{binding} two components is to encapsulate the binding declarations in a separate module, while leaving the bound components untouched. In this study we present a language which satisfies this requirement and offers a reusable, maintainable and concise way of expressing binding. Our language is composed of two main structures; instance pointcuts and adapter declarations. Instance pointcuts are specialized pointcuts, which are used to capture the instances of a type, which at some point in their life cycle become relevant. The relevance of an instance is defined in the pointcut expression. 



Software evolution requires integration of new components with the old ones. In an
ideal component-based system, this integration should be seamless meaning 
 the legacy components remain untouched and the interface of the new component is fully compatible with the existing interfaces. 
Unfortunately such systems do not exist; as a result the integration is seldom
seamless. Throughout the paper we will refer to the integration of two software components as a
\emph{binding}. 

We have defined three major challenges regarding binding. 
\begin{enumerate}
  	\item When components evolve, the links between them
	must be re-established. 
 	 \item When adding unforeseen functionality to a system, no explicit hooks
 	 exist for attaching the new component. 
	\item The interface of the components is not compatible and they should be adapted.
\end{enumerate}

Handling the first challenge requires a \emph{maintainable} way of expressing binding. It is possible to program binding according to some foreseeable evolution scenarios. However in today's component-based systems, third-party software is widely used. So when the interface offered by a third-party software changes, it is necessary to re-program the binding. \emph{Reusable binding structures} and expressing binding in a \emph{concise} manner is then valuable to reduce this programming effort.

Handling the second challenge requires a means to expose certain information in an
application's control flow and inject additional behavior to the control flow. Context exposure can be done via object-oriented programming(OOP), by providing classes which store and expose context information. Injecting additional behavior can be achieved via design patterns like dependency injection or decorator. However these methods are valid for planned extensions, and they will not be sufficient when a new component needs to access an unexposed context. Another issue is linked to the nature of binding two components. Since components need to be connected through possibly multiple points the \textbf{binding concern} becomes cross-cutting. It has been shown that OOP is not effective in modularizing such cross-cutting concerns.  

Handling the third challenge requires a means to express the mapping between components. In this paper we present new language mechanisms to express such mappings and provide improvement on the solutions of the first two challenges. Our approach consists of two language concepts that work together; \emph{instance selectors} and \emph{adapter declarations}. 

In the original GoF book, Adapter Pattern's purpose is defined as ``converting the interface of a class into another interface clients expect. Adapter lets classes work together that could not otherwise because of incompatible interfaces''. Adapters are an important part of a component-based system, since they are the building blocks of binding. 

Our approach is designed as an addition to Aspect-Oriented Programming (AOP)\textbf{(?)}. AOP is used to modularize crosscutting
concerns and with its 'weaving' mechanism, it is possible to change the behavior or the structure of an implementation without altering the
implementation itself. These properties of AOP make it a desirable candidate for modularizing the binding concern.

AOP is effective for achieving loose-coupling. It
can capture information or inject behavior from a component without being
acknowledged. AOP also facilitates the OO way of loose-coupling, which done via
interfaces. It is possible to declare subtypes of an interface and provide the
subsequent implementation in an aspect.  AOP is also efficient in localizing a concern. So when two components are bound using
AOP, the binding implementation will be in one place. This is an important
property for maintaining the modules providing loose-coupling. 

\textcolor[rgb]{0.50,0.50,0.50}{A pointcut is a program construct that selects join points and expose context at
those points \textbf{(AspectJ in Action book)}. Hence by using pointcuts we can
define entry points to a system to inject new behavior. Of course this approach
is limited with the expressiveness of the join-point model of the AO-language. }

However implementation of adapters in current AOP approaches is type invasive. In AspectJ Adapter Pattern is implemented via inter-type declarations which alter the type system by making the adaptee implement a certain interface. This changes the type hierarchy of the adaptee. Also the Adapter Pattern disappears into the AspectJ syntax, which diminishes the visibility of the pattern. CaesarJ has an explicit syntax for defining adapters, which are referred to as wrappers in CaesarJ. However CeaserJ, although it gives great power over separation of concerns, tend to become too fragmented, hurting maintainability. 


\section{The Binding Language}
\subsection{Instance Selectors / Pointcuts}
\subsection{Adapter Declarations}


\section{Comparative Evaluation}
This section will include an example and compare solutions in AspectBind, AspectJ and CaesarJ.



\section{Related Work}


\end{document}

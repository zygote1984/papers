% This is LLNCS.DOC the documentation file of
% the LaTeX2e class from Springer-Verlag
% for Lecture Notes in Computer Science, version 2.4
\documentclass{llncs}
\usepackage{llncsdoc}
\usepackage{color}
\usepackage{graphicx}
\usepackage{listings, tikz, xcolor}
\usepackage{bold-extra}
\usepackage{subfig}
\usepackage{enumerate}
\usepackage[backgroundcolor=pink!40, disable]{todonotes}
\usepackage{courier}
\usepackage{syntax}
\usepackage{amsmath}
\usepackage{hyperref}
%
\lstset{ 
	language = [AspectJ]Java,
  basicstyle=\ttfamily\footnotesize,
  numbers=left,
	numberstyle=\tiny\color[rgb]{0.25,0.25,0.25},
	numberblanklines=false,
  firstnumber=auto,
	breaklines=true,
  tabsize=2,
	emph={aspect,declare, adapter, instance, pointcut, adaptee, adapts, select, UNTIL,pc, instanceType, exp, removeExp, subsetof, supersetof}, 
	emphstyle=\textbf,
	stringstyle=\textsf,
	showstringspaces= false,
	frame=single,
	belowskip = -10,
	captionpos = b,
	numberbychapter=false
	}
	
\newcommand\bh{\tikz[remember picture]
					\node (begin highlight) {};
			 }
  \newcommand\eh{\tikz[remember picture]
		 \node (end highlight) {};
		 \tikz[remember picture, overlay] 
		 \draw[yellow,line width=10pt,opacity=0.3] (begin highlight) -- (end
			highlight);
		 }
	
\begin{document}


\title{Instance Pointcuts}
\subtitle{An Aspect-Oriented Mechanism for Maintaining Object Sets}

\author{Kardelen Hatun \and Christoph Bockisch \and Mehmet Ak\c{s}it}
\institute{TRESE, University of Twente \\7500AE Enschede \\The Netherlands \\
\url{http://www.utwente.nl/ewi/trese/}\\
\email{ \{hatunk,c.m.bockisch,aksit\}@ewi.utwente.nl}
}


\maketitle

\begin{abstract}
In the life-cycle of objects there are different phases. The phase in which an object currently is, affects how it is handled in an application; however phase shifts are typically implicit.
In this study we propose an extension to the aspect-oriented language AspectJ with a new mechanism, called \emph{instance pointcuts}, for creating and maintaining categories of objects according to events in their life-cycle; these events are selected with pointcut-like specifications.


The selection criteria of instance pointcuts can be refined, e.g., by restricting the scope of an existing instance pointcut; and they can be composed, e.g., by boolean operations.
We offer a means to access all objects currently selected by an instance pointcut from Java code, i.e., to be used in methods or advice bodies; and we expose the events of adding or removing an object from an instance pointcut by creating a join point that can be selected by regular pointcuts.
Our approach improves modularity by providing a fine-grained mechanism and a declarative syntax to define and maintain object categories.
\end{abstract}

% A category with the (minimum) three required fields
\category{D.3.1}{Formal Definition and Theory}[syntax, semantics]
%A category including the fourth, optional field follows...
\category{D.3.4}{Processors}[code generation]

\section{Introduction}
In object-oriented programming (OOP), objects encapsulate state and behavior; objects also have a life-cycle, which means that the same object can play different roles at different times.
Which role an object is currently playing can affect the object's own behavior or how it is handled.
Typically the shift from one life-cycle phase to another is implicitly marked by events, e.g., passing an object from one client to another.

As an example of relevant phases in the life-cycle of objects, consider an online store application with ``vendor'' objects representing the suppliers and ``product'' objects representing the products they sell. 
Assume we would like to keep the list of products which were applied the happy-hour discount. Each product is assigned to a different time slot as its happy-hour, so the list of products that are discounted is changing over time. 
Categorizing objects according to criteria that is not directly supported by the programming language, such as which class they were initialized in, which method they were passed to as an argument, or (as in the example) the time at which they are passed to a method, requires invasively inserting bookkeeping code.

Aspect-oriented programming can be applied to separate this bookkeeping code from the business logic of the program. But in AOP, \emph{pointcuts} select sets of so-called \emph{join points} which are points in time during the execution of the program. Current aspect-oriented languages do not support a \emph{declarative specification} of the objects belonging to a life-cycle phase; instead an \emph{imperative implementation}, always following the same pattern, is required for collecting those objects.
A consequence of such an imperative solution is reduced readability and maintainability due to boilerplate code.


To offer better support for processing objects according to their life-cycle phase, we propose a new mechanism, called \emph{instance pointcuts}, to select sets of objects based on the events in their execution history.
Instance pointcuts are used to declare the beginning and the end of a life cycle as events. Instance pointcuts can be reused via refinement and composition operations and their implementation provides checks to ensure correctness. 

An instance pointcut's concise definition consists of three parts: an identifier, a type which is the upper bound for all objects in the selected set, and a specification of relevant objects.
The specification utilizes \emph{pointcut expressions} to select events that define the begin and end of life-cycle phases and to expose the object. At these events, an object is added or removed from the set representing the instance pointcut. 
New instance pointcuts can be derived from existing ones. Firstly, a new instance pointcut can be derived from another one by restricting the type of selected objects. 
Secondly, instance pointcut declarations can be composed arbitrarily by means of set operators. 
The declarative nature of instance pointcuts allow several checks to be performed which are not automatically possible with equivalent imperative code. Such checks are important to notify the developer about empty sets, ill-formed composition expressions,  illegal instance pointcut compositions and refinements, etc.  These checks ensure the correctness of the instance pointcuts and consistence in object categories. 

In this paper we present instance pointcuts as an extension to AspectJ \cite{kiczales2001overview} and explain its semantics by explaining our compiler which transforms instance pointcuts to plain AspectJ and advanced dispatching library calls. 

The rest of the paper is organized as follows, in section~\ref{sect:motivation} we present a small case study and explain our motivation for the proposed approach. In section~\ref{sect:ip} a detailed description of instance pointcuts and its various features are presented. This section also explains how instance pointcuts are compiled. We then present a discussion on the validation of our approach. We conclude by discussing related work and giving a summary of our approach.


\section{Motivation}
\label{sect:motivation}

Objects can be categorized by how they are used (passed as arguments to method calls, act as receiver or sender for method calls, etc.) and concerns of an application may be applicable only to objects used in a specific way.
Therefore we must be able to identify and select those objects.
We want to expose sets of objects belonging to the same category by means of a dedicated language construct such that the implementation of phase-dependent concerns can explicitly refer to the category.

\begin{figure*}
\centering
\includegraphics[width= 0.75\textwidth]{images/onlineshop2.png}%
\vspace{10pt}
\caption{Part of an online shop application}%
\label{fig:shop}%
\end{figure*}

In Figure~\ref{fig:shop}, we outline a part of the architecture of an online store application from the \emph{system's perspective}. We use this scenario to give examples of categorizing objects according to how they are used and how to use these categories in the implementation of concerns. It must be noted that we intend to support the addition of \emph{unanticipated} concerns, i.e., the program code is not prepared to support abstractions, like specific object categories, required by the new concerns. 
%At the end of this section, we conclude requirements for solving the encountered challenges in these examples.

\subsection{Example Architecture}
An online shop is a sophisticated web application and objects of the same type can exist at different stages of their life-cycle. In Figure~\ref{fig:shop} the static structure of a simplified online shop is shown. This structure shows part of the system from the \lstinln{Vendor} and the \lstinln{OnlineShop}'s perspective. \lstinln{Vendor}s can submit different kinds of \lstinln{Discount}s to the \lstinln{ProductManager} for the \lstinln{Product}s they are selling. \lstinln{Product} is the root of the type hierarcy and is parent to the types such as \lstinln{BeautyProduct}, \lstinln{SportProduct}~\dots(Not shown in the figure). Each \lstinln{Product} holds a list of \lstinln{Discounts} they are applied. The \lstinln{OnlineShop} has a user interface represented by the \lstinln{OnlineShopUI} class, which is used to display information to the customers. 


\subsection{Unanticipated Extensions}
A new feature is added to the online shop which requires creating an alert when a product is applied a surprise discount, also the list of surprise discounted products should be available to the user at any time. The surprise discounts are submitted by \lstinln{Vendor}s and they can be submitted or withdrawn any time.  In order to realize this extension in an OO-approach, we need to change several classes to host this extension. First the class \lstinln{ProductManager} should keep a set of \lstinln{Product}s which are applied a surprise discount, in Listing~\ref{lst:discountalert} this is shown in line~\ref{surpdisset}. This set is updated when a new discount of type \lstinln{SurpriseDiscount} is submitted or withdrawn (lines~\ref{surpset:begin}--~\ref{surpset:end}). There should also be some changes in the \lstinln{OnlineShop} class. First a \lstinln{createDiscountAlert} method should be added. Also the \lstinln{displayList} method should be updated to include the \lstinln{surpriseDiscount} list defined in the \lstinln{ProductManager} class.


\begin{lstlisting}[float, caption={A Java implementation of discount alert concern}, label={lst:discountalert}]
class ProductManager{
	...
	~\bh~Set<Product> surpriseDiscount = createSet(); ~\eh\label{surpdisset}~
	public void submitDiscount(Product p, Discount d)~\label{surpset:begin}~
	{
		...
		if(d instanceof SurpriseDiscount)
		{
			surpriseDiscount.add(p);
			OnlineShop.createDiscountAlert(p);
		}
	}
	public boolean withdrawDiscount(Product p, Discount d)
	{
		...
		if(d instanceof SurpriseDiscount)
			surpriseDiscount.remove(p);
	}~\label{surpset:end}~
}
class OnlineShop{//SINGLETON
	...
	public void createDiscountAlert(Product p)
	{
		//create surprise discount alert for p
	}
	public void displayList(String listType)
	{
		if(listType.equals(``surprise'')
			INSTANCE.getUI().display(ProductManager.surpriseDiscount);
	}
}
\end{lstlisting}


The OO-solution is scattered among \lstinln{ProductManager} and \lstinln{OnlineShop} classes and tangled with multiple methods. The code for the surprise discount concern and the book-keeping that comes with it creates cluttering. 
An aspect-oriented implementation can offer a better solution by encapsulating the concern in an aspect. 
In Listing~\ref{lst:discountaop} shows a possible solution. The set of products which are applied a surprise discount is kept in the aspect (line~\ref{daop:set}). The following two pointcuts \lstinln{submit} and \lstinln{withdraw} selects the products to which a \lstinln{SurpriseDiscount} is applied (lines~\ref{daop:pc:submit}--~\ref{daop:pc:wdraw}). The corresponding advice declarations for these pointcuts maintain the \lstinln{surpriseDiscount} set, also the \lstinln{submit} join-point triggers the surprise discount alert method (line~\ref{daop:alert}). There is also the \lstinln{display} pointcut (line~\ref{daop:pc:display}, which intercepts the call to \lstinln{displayList} method and add the condition for the surprise discount list in an around advice (lines~\ref{daop:around:begin}--~\ref{daop:around:end}). This aspect also includes an inter-type declaration which adds the \lstinln{createDiscountAlert} method to the \lstinln{OnlineShop} class.


\begin{lstlisting}[float=h, caption={An Aspectj implementation of discount alert concern}, label={lst:discountaop}]
~\hilight~Set<Item> surpriseDiscount = createSet();~\label{daop:set}~
~\hilight~pointcut submit(Product p): call(* ProductManager.submitDiscount(..)) && args(p, SurpriseDiscount); ~\label{daop:pc:submit}~
~\hilight~pointcut withdraw(Product p): call(* ProductManager.withdrawDiscount(..)) && args(p, SurpriseDiscount); ~\label{daop:pc:wdraw}~
pointcut display(String listType): call(* OnlineShop.displayList(..)) && args(listType);~\label{daop:pc:display}~
after(Product p): submit(p){
	~\hilight~surpriseDiscount.add(p);
	OnlineShop.instance().createDiscountAlert(p);~\label{daop:alert}~
}
after(Product p):withdraw(p){
	~\hilight~if(surpriseDiscount.contains(p))
		~\hilight~supriseDiscount.remove(p);}
}
void around(String listType):display(listType){~\label{daop:around:begin}~
	if(listType.equals(``surprise'')
		OnlineShop.instance().getUI().display(surpriseDiscount);
	proceed(listType);	
} ~\label{daop:around:end}~
public void OnlineShop.createDiscountAlert(Product p){
	//create surprise discount alert for a product
}
\end{lstlisting}

\paragraph{Discussion and Requirements}

AOP already helps to localize the concern and to add it without the need to modify existing code.
%But most of the implementation of the surprise discount concern as shown in Listing~\ref{lst:discountaop} consists of boilerplate code .
However maintenance of the \lstinln{surpriseDiscount} set requires the same boilerplate code the OO solution does (highlighted in gray).
Essentially what the code does is select \lstinln{Product} objects based on the discount they are applied to and deselect them once they are rid of this discount. 
This marks a phase in the life-cycle of a \lstinln{Product} object, none of the presented solutions offer any declarative means to define such a life-cycle phase.
Furthermore, reusing such existing sets by refining or composing them is not conveniently supported at all; e.g., if we want to find the subset of \lstinln{BeautyProduct}s of the \lstinln{surpriseDiscount} set, we have to iterate over it and check instance types to create a new set. This iterative way of refining does not give us any informationabout the set being empty or populated; this kind of a check also needs additional code. From this motivating example we conclude our problem statement as follows.

\paragraph{Problem Statement}
Creating objects sets according to execution events is a crosscutting concern. Current programming approaches only provide imperative ways to maintain object sets which result in code that is marginally concern-specific and majorly boilerplate. The maintenance costs of such programs hinder software evolution and their imperative nature prevents useful checks.







%\begin{enumerate}[{Requirement}1{:}]
%\item A declarative way of reifying a set of objects by defining add/remove conditional expressions must be provided.
%\item Since objects may recursively enter and exit a life-cycle phase, it must be counted how often the begin and end events have occurred for each object.
%\item Add/Remove expressions must select events and specify which object from the context of the event is selected.
%\item It must be possible to access the set of objects which currently comprises an object category and to be notified when the set changes.
%\item A declarative way to obtain new sets by composing existing ones is required.
%\item Validity should be ensured for composition, either by construction or through compile-time checks.
%\end{enumerate}


\section{Instance Pointcuts}
Instance pointcut is a declarative language construct that is used to reify and maintain a set of objects of a specific type, with the ability to select them over a period marked by events in their life-cycle, modularizing the object selection concern. The purpose of instance pointcuts is to  let the user to make focused selections, therefore offering a different level of modularity. 
 
In the remainder of this section, we will explain instance pointcuts in detail. We have implemented a prototype by extending AspectJ, throughout the section code examples of instance pointcuts will be in AspectJ.

%\todo[inline]{maybe section outline summarizing main features}
%\missingfigure{Static Structure of instance pointcuts}

\subsection{Features}

A concrete instance pointcut definition consists of a left hand-side and a right-hand side. 
In the left hand side the pointcut's name and the instance type of interest is declared. 
An instance pointcuts do not declare pointcut parameters, it has a single implicit parameter called \texttt{instance} of the declared instance type. 
In the right hand side a pointcut expression selects the desired events from join-points and then binds the exposed object as a member of the instance pointcut's set, which is represented by \texttt{instance} parameter. 
An instance pointcut is a \emph{static aspect member} (Listing \ref{lst:member}).%, can be accessed as a field;  \lstinline!MyAspect.foo!.

\begin{lstlisting}[float=h!, caption={An instance pointcut declaration in an aspect}, label={lst:member}]
aspect MyAspect{
	static instance pointcut foo<Object> ...
}
\end{lstlisting}
%It is also possible to declare an abstract instance pointcut, by leaving out the right hand side and placing the \emph{abstract} modifier at the beginning of the declaration and not defining a pointcut expression. Abstract instance pointcuts are used when execution details are not yet clear but the instance set to be created and its instance type are determined.

\subsubsection{Add/Remove Expressions}
Join points mark \emph{sites} of execution; a join-point by itself does not define an event. 
In AspectJ pointcuts, pointcut expressions select joinpoints and pointcuts are used with advice specifications to select a particular event in that join-point.
Instance pointcuts are not used with advices; since they serve the specific purpose of reifying and maintaining object sets, they implicitly perform add to set and remove from set operations. 
Therefore instance pointcut expressions are used to select specific events rather than joinpoints.
To be able to define events with instance pointcut expressions, it is necessary to use advice specification clauses \texttt{before} or \texttt{after} in the expression. 

The \emph{add expression} select events which contain the instances to be added to the instance pointcut set; an event selected by add expression marks the beginning of the life-cycle phase of interest. 
Instance pointcuts can also include an optional \texttt{remove expression},  which specifies when to remove which object, selects the event that marks the end of the life-cycle phase of interest.
Both add and remove expressions contains at least one \emph{expression element} and at most two.
The expression elements can be one of two types, \emph{before element} and an \emph{after element}.
Each element contains a pointcut expression and they can select more than one event. 
The before element selects the before events and the after element, selects after events. 

The grammar definition below shows part of the instance pointcut syntax. 
An event definition is comprised of an event keyword , \texttt{before} or \texttt{after}, and a pointcut expression.
The pointcut expression selects a set of join points and the event keyword specifies which event is selected from a join-point. 
\todo{is it sufficient?} $\langle pointcut-expression\rangle$ rule represents an AspectJ pointcut expression with the restriction that every pointcut expression there should be a binding predicate that bind the \texttt{instance} parameter and an extension that allows the use of \texttt{returning} clause as a binding predicate. 
In an instance pointcut expression, it is only possible to \texttt{OR} a before event with an after event. 
The \emph{before} clause selects the start of executing an operation (i.e., the start of a join point in AspectJ terminology) and the \emph{after} clause selects the end of such an execution. 
For two operations that are executed sequentially, the end of the first and the start of the second operation are treated as two different events. Thus, the before and after clauses select from two disjoint groups of events and the conjunction of a before and an after clause will always be empty.


\begin{figure}[h]
\begin{grammar}
<instance pointcut> ::= `instace pointcut' <name> `<' <instance-type> `>' `:'
<ip-expr> (`UNTIL' <ip-expr>)? 

<ip-expr> ::= <after-event> `||' <before-event> \alt <before-event> `||' <after-event> \alt <after-event> \alt <before-event>

<after-event> ::= `after' `('<pointcut-expression>`)'

<before-event> ::= `before' `('<pointcut-expression>`)'
\end{grammar}
\caption{Grammar definition for instance pointcuts}
\label{fig:grammar1}
\end{figure}


\begin{lstlisting}[float=h!, caption={A basic instance pointcut declaration with add and remove expressions}, label={lst:basic}]
instance pointcut customers<Customer>: after(call(SessionManager.createCustomer(..)) && returning(instance)) UNTIL before(call(SessionManager.destroyCustomer(..)) && args(instance))
\end{lstlisting}


The instance pointcut in Listing \ref{lst:basic} shows a basic example. The left-hand side of the instance pointcut indicates the pointcut is called \texttt{customers} and it is interested in selecting the \texttt{Customer} objects. 
On the right hand side, there are two expression separated by the \texttt{UNTIL} keyword. The first one is the add expression. It selects the return event of the method \texttt{createCustomer} and from the context of this event it exposes the returned \texttt{Customer} object with the \texttt{returning} clause and binds it to the \texttt{instance} parameter. If the \texttt{instance} parameter is bound by the returning clause then it has to be in an after event. The second one is the remove expression and it selects the before event \texttt{destroyCustomer} call and exposes the \texttt{Customer} instance in the method arguments and binds it to the \texttt{instance} parameter with \texttt{args} pointcut.

\emph{Note that instance pointcuts do not keep objects alive.} Because instance pointcuts are non-invasive constructs, which do not affect the program execution in any way. So even if the remove expression was not defined for \texttt{customers} pointcut, if the \texttt{Customer} instances were collected by the garbage collector, they would still be removed from the set. 


\subsubsection{Multisets}
An instance pointcut reifies an object set as a \emph{multiset}. Multisets are sets which allow multiple appearances of a an object. The instance pointcut shown in In Listing \ref{lst:wishlisted} selects \texttt{Product} instances, which are added to the wishlist of a \texttt{Customer}. The remove expression removes a \texttt{Product} instance if it is removed from the wish-list of a \texttt{Customer}. With this pointcut we would like to count the currently wish-listed products. If instance pointcuts onyle supported a set then as soon as a customer removes a product from his wish-list, its only copy will be removed and it will appear removed from all customers' wish-lists, therefore a multi-set is used. 

\begin{lstlisting}[float=h!, caption={An instance pointcut utilizing multiset property}, label={lst:wishlisted}]
instance pointcut multi_wish<Product>: 
before(call(Customer.addToWishList(..)) && args(instance))
 UNTIL before(call(Customer.removeFromWishList(..)) && args(instance))
\end{lstlisting}


\subsection{Refinement and Composition}

Instance pointcuts can be referenced by other instance pointcuts, they can be refined to create a subset or broadened to create a superset. 

\subsubsection{Referencing and Dynamic Refinement}
Normally instance pointcuts are referenced by their name, however they can also take an additional statement for \emph{type refinement}, which selects a subset of the instance pointcut \emph{dynamically}. Type refinements require that, the refined type is a subtype of the original instance type. For example the instance pointcut \texttt{lowSelling} (Listing \ref{lst:ref1}) can be refined with the following syntax: \lstinline!instance pointcut lowSellingBeauty<BeautyProduct>: lowSelling<BeautyProduct>!. 
This indicates we would like to select the subset of \texttt{BeautyProduct} instances from the set of \texttt{Product} instances selected by \texttt{lowSelling}. 
Note that this notation will also select subtypes of \texttt{BeautyProduct}. 
The effect of a refinement subset being empty is equivalent to that of an unmatched pointcut, meaning it will not result in any changes on the set.


\begin{lstlisting}[float=h!, caption={Two instance pointcuts composed to obtain a new one}, label={lst:ref1}]
instance pointcut lowSelling<Product>: after(call(* OnlineShop.setLowSelling(..))  && args(instance));
\end{lstlisting}

\subsubsection{Instance Pointcut Composition}
It is also possible to compose instance pointcuts to obtain a new pointcut based on others. In Figure \ref{fig:grammar2}, an extended version of the grammar definition is shown. A new pointcut can be defined either by referencing another pointcut and appending additional predicates to the pointcut expressions in before and after blocks or it can be defined as the composition of other pointcuts or both. In order to obtain a more refined set \&\& operation is used and for a broader set $\|$ is used. 


\begin{figure}[h]
\begin{grammar}
<instance-pointcut> ::= `instace pointcut' <name> `<' <instance-type> `>' `:'
(<ip-ref-expr> <logical-operator>)? (<ip-expr> `UNTIL' <ip-expr>)? 

<logical-operator> ::= `&&' \alt `||'

<ip-ref-expr> ::= <ip-ref> \alt <ip-ref-expr> `&&' <ip-ref-expr> \alt <ip-ref-expr> `||' <ip-ref-expr>

<ip-ref> = <name> \alt <name>(`<' <refined-instance-type> `>')?

\ldots
\end{grammar}
\caption{Syntax for instance pointcut composition}
\label{fig:grammar2}
\end{figure}


The composition also has an effect on the types of the instances that are added to the set. To better explain this effect we have created a small example, where a members of a simple type hierarchy (Figure \ref{fig:types}) are used in instance pointcuts. Table \ref{table1} shows four distinct cases. In the first case we compose two pointcuts \texttt{pc1<B>} and \texttt{pc2<C>} with an \&\& operation, the resulting pointcut from this composition can only select the narrower type \texttt{C} since the composed expression is stricter and the type restriction comes from \texttt{pc2}. However with $\|$ operation we can capture \texttt{B} types as well as \texttt{C} types, so the resulting compsition will select a mixture. In the second row a case with sibling types are shown. For the \&\& operation the resulting composition cannot select any types since the types \texttt{A} and \texttt{B} cannot have a common instance. The $\|$ operation will again select a mix of instances of type \texttt{A} and \texttt{B}, then the instance type for the composed pointcut should be \texttt{R} to contain the subtypes.

\begin{figure}[h!]
\centering
\subfloat[A simple type hierarchy]{
\includegraphics[width=0.1\textwidth]{images/types.png}
\label{fig:types}
}
\hspace{20}
\subfloat[Pointcut compositions and effect on the captured instance type]{ 
\label{table1}
\centering
\begin{tabular}[b]{|p{1cm}|p{1cm}|p{1cm}|p{1cm}|}
\hline 
pc1 type & pc2 type     &\&\&         &$\|$  \\[1ex]
\hline
B & C & C & B \\[1ex]
\hline
B & A & \emptyset & R\\[1ex]
\hline 
 \end{tabular}
}
\caption{An example to illustrate composition's effect on types}
\label{fig:compotypes}
\end{figure}


\paragraph{Semantics}
The composition of two instance pointcuts is computed by composing their instance pointcut expressions.
An instance pointcut can have at most four expression elements; before (part of) add ($ba$) and after add ($aa$) form the add expression and, before remove($br$) and after remove($ar$) form the remove expression. Let us define an instance pointcut as $IP = ba_{IP}\;\|\;aa_{IP}\;\textrm{UNTIL}\;br_{IP}\;\| ar_{IP}$. Each element contains a pointcut expression referred with the following notation; $ba_{IP}\_expression$. 
\begin{align}
	IP1 = ba_{IP1}\;\|\;aa_{IP1}\;\textrm{UNTIL}\;br_{IP1}\;\| ar_{IP1} 
\label{i1}	\\
	IP2 = ba_{IP2}\;\|\;aa_{IP2}\;\textrm{UNTIL}\;br_{IP2}\;\| ar_{IP2}
\label{i2}
\end{align}

Let us define a new pointcut as; $IP3 = IP1\;\textrm{\&\&}\;IP2$. By distributivity this operation is results in the following formulation:

\begin{align}
\begin{split}
IP3 \rightarrow \textrm{add expression} =& \left( ba_{IP1}\;\textrm{\&\&}\;ba_{IP2}\right)\;\\ 
							&\|\;\left( ba_{IP1}\;\textrm{\&\&}\;aa_{IP2} \right)\;\\
&\|\;\left( aa_{IP1}\;\textrm{\&\&}\;aa_{IP2} \right)\;\\
&\|\;\left( aa_{IP1}\;\textrm{\&\&}\;ba_{IP2} \right) 
\label{i3add}
\end{split}
\\
\begin{split}
IP3 \rightarrow \textrm{remove expression}  =& \left( br_{IP1}\;\|\;br_{IP2}\right)\; \\
&\|\;\left( br_{IP1}\;\|\;ar_{IP2} \right)\;\\
&\|\;\left( ar_{IP1}\;\|\;ar_{IP2} \right)\;\\
&\|\;\left( ar_{IP1}\;\|\;br_{IP2} \right) 
\label{i3remove}
\end{split}
\end{align}

Notice that the remove expressions are applied $\|$ instead of \&\&. If we narrow down removal criteria by \&\& then it is possible that some objects which are added will not be removed. By keeping the removal criteria for both of the composed pointcuts, we ensure that all the added object can be removed. So we can say that the \&\& and the $\|$ operations only effect the expression elements in the add expression.

Composition of before and after event via \&\& will be empty as explained at the beginning of this section. Composing two events of the same type with a logical operator is formulated as follows:
\begin{align*}
event(exp_1)\;\&\&\;event(exp_2) = event(exp_1\;\&\&\;exp_2) \\
event(exp_1)\;\|\;event(exp_2) = event(exp_1\;\|\;exp_2)
\end{align*}

Note that we do not evaluate expressions to verify their composition and leave that to the AspectJ compiler. For example AspectJ does not allow use of different binding predicates in the composed expression and this is also a limitation of instance pointcut composition. 

Then IP3 becomes;
\begin{align}
\begin{split}
IP3 =& \overbrace{\left( ba_{IP1}\;\textrm{\&\&}\;ba_{IP2}\right)}^{ba_{IP3}}\;\|\;\overbrace{\left( aa_{IP1}\;\textrm{\&\&}\;aa_{IP2} \right)}^{aa_{IP3}}\;\\
&\textrm{UNTIL} \\
&\underbrace{\left( br_{IP1}\;\|\;br_{IP2}\right)}_{br_{IP3}}\;\|\;\underbrace{\left( ar_{IP1}\;\|\;ar_{IP2} \right)}_{ar_{IP3}}\;\\  
\label{ip3step}
\end{split}
\end{align}

If we define another instance pointcut as $ IP4 = IP1\;\|\;IP2$, after performing the above formulations:
\begin{align}
\begin{split}
IP4 =& \overbrace{\left( ba_{IP1}\;\|\;ba_{IP2}\right)}^{ba_{IP4}}\;\|\;\overbrace{\left( aa_{IP1}\;\|\;aa_{IP2} \right)}^{aa_{IP4}}\;\\
&\textrm{UNTIL} \\
&\underbrace{\left( br_{IP1}\;\|\;br_{IP2}\right)}_{br_{IP4}}\;\|\;\underbrace{\left( ar_{IP1}\;\|\;ar_{IP2} \right)}_{ar_{IP4}}\;\\  
\label{ip3step}
\end{split}
\end{align}

Using the same composition principles, instance pointcuts can also be composed with expression elements, to refine or broaden the pointcut scope. For the instance pointcut IP5 which is defined as:

\begin{align}
\begin{split}
IP5 & = IP1\;||\;(ba_{new}\;UNTIL\;ar_{new}) \\
   & = \overbrace{\left( ba_{IP1}\;\|\;ba_{new}\right)}^{ba_{IP5}}\;\|\;\overbrace{\left( aa_{IP1} \right)}^{aa_{IP5}}\;\\
&\textrm{UNTIL} \\
&\underbrace{\left( br_{IP1}\right)}_{br_{IP5}}\;\|\;\underbrace{\left( ar_{IP1}\;\|\;ar_{new} \right)}_{ar_{IP5}}\;\\  
\end{split}
\end{align}

Let us illustrate these concepts with a concrete example. The first instance pointcut  (\texttt{lowSelling})  shown in Listing \ref{lst:refmain} selects the \texttt{Product}s which are marked as \emph{low-selling} by the \texttt{OnlineShop} class and remove those products which are discontinued by the \texttt{Vendor}. The second pointcut (\texttt{wishList})  selects the \texttt{BookProduct}s which are added to the wish-list of a \texttt{Customer}, removes them when they are out of the wishlist.

\begin{lstlisting}[float=h!, caption={Two instance pointcuts composed to obtain a new one}, label={lst:refmain}]
instance pointcut lowSelling<Product>: after(call(* OnlineShop.setLowSelling(..))  && args(instance)) UNTIL before(call(* Vendor.discontinue(..)) && args(instance));
instance pointcut wishList<BookProduct>: after(call(* Customer.addToWishList(..)) && args(instance)) UNTIL before(call(* Customer.removeFromWishList(..)) && args(instance)) ;
instance pointcut composition<Product>: lowSelling || purchase;
\end{lstlisting}

The new instance pointcut \texttt{composition} is the result of an $\|$ operation between \texttt{lowSelling} and \texttt{wishList}. Then the add and remove expressions of \texttt{composition} is created by factoring out the \texttt{args} pointcut and is shown in Listing \ref{lst:newpc}. 

\begin{lstlisting}[float=h!, caption={The new instance pointcut after composition}, label={lst:newpc}]
instance pointcut composition<Product>: 
after((call(* OnlineShop.setLowSelling(..)) || call(* Customer.addToWishList(..))) && args(instance)) 
UNTIL 
before((call(* Vendor.discontinue(..)) || call(* Customer.removeFromWishList(..))) && args(instance))
\end{lstlisting}


Similarly we can define a new pointcut (Listing \ref{lst:comp2}), this time by composing a pointcut with an expression element, then the \texttt{composition2} pointcut has the pointcut expression shown in Listing \ref{lst:comp2exp}.

\begin{lstlisting}[float=h!, caption={An instance pointcuts and an expression is composed}, label={lst:comp2}]
instance pointcut composition2<Product>: lowSelling && (after(if(instance.getPrice() > 10)) UNTIL before(call(* OnlineShop.removeFromShop(..)) && args(instance));
\end{lstlisting}


\begin{lstlisting}[float=h!, caption={composition2 pointcut expression in full}, label={lst:comp2}]
instance pointcut composition2<Product>: 
(call(* OnlineShop.setLowSelling(..))  && args(instance) && if(instance.getPrice() > 10)) 
UNTIL
before((call(* Vendor.discontinue(..)) || call(* OnlineShop.removeFromShop(..))) && args(instance));
\end{lstlisting}



\subsubsection{Subset and Superset Relationships}
There are two mechanism to define relationships between instance pointcuts, \emph{subsetting} and \emph{supersetting}. The declaration of these relationships impose some restrictions on the \emph{declaring} instance pointcut. Let us update the syntax definition to host relationship declarations.

\begin{figure}[h]
\begin{grammar}
<instance-pointcut> ::= \ldots \alt `instace pointcut' <name> (`<' <instance-type> `>')? <relationship> `:' <ip-expr> (`UNTIL' <ip-expr>)?  \alt
	`instace pointcut' <name> (`<' <instance-type> `>')? <relationship> `:' <ip-ref>

<relationship> ::= `subsetof' <ip-ref> \alt `supersetof' <ip-ref>
\end{grammar}
\caption{Grammar definition for relationship declarations}
\label{fig:grammar3}
\end{figure}

For the following declaration \lstinline!subsetIP1 subsetof IP1!, the subset relationship imposes that the instance set created by \texttt{subsetIP1} has to be the subset of the instance set of \texttt{IP1}. Hence, if later in the implementation, the scope of \texttt{subsetIP1} is broadened, it should still follow the subset restriction. When the\emph{instance type} is not declared like in the example then it is inferred from the superset \texttt{IP1}. It is also possible to declare subsets by using type refinement. For example \lstinline!subsetIP1<SubType> subsetof IP1 = IP1<SubType>!, creates a subset of \texttt{IP1} by selecting the instances of type \texttt{SubType}. A subset must declare a co-variant type of its superset and superset must declare a contra-variant type.
For the superset declaration \lstinline!supersetIP1 supersetof IP1!, the superset relationship imposes that the instance set created by \texttt{supersetIP1} has to contain the set of \texttt{IP1}. This rule should hold even if the scope of \texttt{supersetIP1} is narrowed. When the instance type is not declared then the instance type of \texttt{IP1} is copied. 

The right-hand side of a relationship declaration can be a full instance pointcut expression, formed by add and remove pointcuts. For subsets, \&\& instance pointcut composition rule is applied, composing this expression with the superset's and for supersets, $\|$ composition rule is applied to the subset's, composing the expression with the subset's. 

Subset and superset relationships enforce consistency between instance pointcuts and reduce redundancy. Being able to select subsets/supersets of instance pointcuts, eliminates the need for defining separate pointcuts, which may result in erroneous selections. Since the subsets only supports the subtypes of the original instance type and superset support only the supertypes, it works with the system's type hierarchy in a natural manner. 

\todo[inline]{example}

\subsection{Using Instance Pointcuts}
Upto now we have explained the syntax and semantics for instance pointcuts. In this section we will explain how to use instance pointcut in the context of an AO languge, namely, AspectJ. The instance pointcut defined in Listing \ref{lst:usage} satisfies the following requirement; \textsf{Maintain a set which contains products that are currently out of stock, once they become available remove them from the set}.

\begin{lstlisting}[float=h!, caption={An instance pointcut for out of stock products}, label={lst:usage}]
aspect MyAspect{
	static instance pointcut outOfStock<Product>: after(call(* Product.outOfStock(..)) && target(instance)) UNTIL after(call(* Vendor.stock(..)) && args(instance));
	...
}
\end{lstlisting}

\paragraph{Set Access}
Instance pointcuts reify a set and this set can be accessed through the pointcut. When a pointcut is referenced outside of a pointcut expression (composition, subset etc.), it means that its set is accessed. Then the \emph{read} methods of the multiset\textbf{(google guava?)} interface can be used to retrieve objects from the set. Write methods are not allowed since, it creates data inconsistencies and in may result in concurrent modification exceptions. The set can also be referenced with a type refinement, which will return a subset of the original set with objects of refined type. 

\begin{lstlisting}[float=h!, caption={Calculate a damage estimate for out of stock beauty products}, label={lst:setaccess}]
public static double calculateDamages()
{
	double damage = 0;
	for(BeautyProduct p: MyAspect.outOfStock<BeautyProduct>)
		damage = damage + p.getPrice();
	return damage;
}
\end{lstlisting}

In Listing \ref{lst:setaccess},  a method called \texttt{calculateDamages} is defined \texttt{MyAspect}. This method calculates a rough estimate of the shop's damages when beauty products are out of stock. On line 4, the for loop iterates over the \texttt{BeautyProduct} subset of \texttt{outOfStock}'s set, which is accessed as a static field and also with the refinement syntax.


\paragraph{Set Monitoring}
An instance pointcuts definition defines two set change events; add event and remove event. These events are interesting since they signify that the object has become relevant or irrelevant. In order to select the join-points of these events, every instance pointcut definition automatically has two implicit regular pointcuts. These implicit pointcuts has the following naming convention; $\langle name \rangle\_add, \langle name \rangle\_remove$, where $\langle name \rangle$ is the name of the instance pointcut. These pointcuts allow the user to access the set change event and the object to be added or removed. In Listing \ref{lst:monitor1} a before advice using the \texttt{outOfStock_add} pointcut is shown. When a product is marked out of stock then it is added to the set, this advice uses the \texttt{Product} instance to be added and uses its information to notify the related \texttt{Vendor} that the product is out of stock.

\begin{lstlisting}[float=h!, caption={Set monitoring pointcut used to notify vendors}, label={lst:monitor1}]
before(Product p): outOfStock_add(p)
{
	OnlineShop.notifyVendor(p.getVendor, STOCK_MSG);
}
\end{lstlisting}

\subsection{Compilation of Instance Pointcuts}
In our current implementation, we compile instance pointcuts to AspectJ code. For the instance pointcut shown in Listing \ref{lst:usage} and the code snippets shown in Listings \ref{lst:setaccess} and \ref{lst:monitor1}, the AspectJ code shown in Listing \ref{lst:generated} is generated. 

The lines 2-27 are generated from the instance pointcut \texttt{outOfStock}. On line 2, a multiset, named \texttt{outOfStock_set} with the generics \texttt{Product} is instantiated. The generic type is inferred from the instance type declared in \texttt{outOfStock}. This particular instance pointcut has only two expression elements; an after element in the add expression and an after element in the remove expression. These expression elements result in two pointcuts shown in lines 4 and 5 and two advice bodies shown in lines 10-14 and 15-19 respectively. For adding elements to the set, \texttt{add_outOfStock_set} method (lines 20-23) and for removing \texttt{remove_outOfStock_set}(lines 24-27) are generated. The set monitoring pointcuts for these methods are \texttt{outOfStock_add(Product p)} and \texttt{outOfStock_remove(Product p)}, respectively.



\begin{lstlisting}[float=h!, caption={The generated AspectJ code from \texttt{outOfStock} instance pointcut}, label={lst:generated}, escapechar=@]
aspect MyAspect{
	public static Multiset<Product> outOfStock_set =
			new TreeMultiset<Product>();
	pointcut outOfStock_aa(): call(* Product.outOfStock(..));
	pointcut outOfStock_ar(): call(* Vendor.stock(..));
	pointcut @\underline{outOfStock_add(Product p)}@: 
		call(* MyAspect.add_outOfStock_set(..)) && within(MyAspect) && args(p);
	pointcut outOfStock_remove(Product p): 
		call(* MyAspect.remove_outOfStock_set(..)) && within(MyAspect) && args(p);
	after():outOfStock_aa()
	{
		Product p = thisJoinPoint.getTarget();
		add_outOfStock_set(p);
	}
	after():outOfStock_ar()
	{
		Product p = findArgument(thisJoinPoint.getArgs(),Product.class);
		remove_outOfStock_set(p);
	}
	public boolean add_outOfStock_set(Product p)
	{
		return outOfStock_set.put(thisJoinPoint.getTarget());
	}
	public boolean remove_outOfStock_set(Product p)
	{
		return outOfStock_set.remove(p);
	}
	public static double calculateDamages()
	{
		Multiset<BeautyProduct> @\underline{outOfStock_BeautyProduct_set}@ = new TreeMultiset<BeautyProduct>();
		for(Product p: outOfStock_set)
			if(p instanceof BeautyProduct)
				outOfStock_BeautyProduct_set.put(p)
		double damage = 0;
		for(BeautyProduct p:outOfStock_BeautyProduct_set)
			damage = damage + p.getPrice();
		return damage;
	}
	before(Product p):@\underline{outOfStock_add(p)}@
	{
		OnlineShop.notifyVendor(p.getVendor, STOCK_MSG);
	}
}
\end{lstlisting}


\section{Related Work}
\section{Discussion}
\section{Conclusion and Future Work}
\todo[inline]{keep it short}

\end{document}


%\subsection{Potential Scenarios}
%\begin{enumerate}
	%\item When a product is wish-listed by a GoldCustomer, then apply a discount to it. 
	%\item When products are added to a shopping bag between certain hours, then they are applied the happy-hour discount
	%\item When products are added to GoldCustomer’s shopping bags, apply an additional discount
	%\item When a product is marked low selling by the system, put an advertisement for it on the sidebar → refinement low selling books..., 
	%\item if product is low selling →  notify the vendor
	%\item superset: beauty products, sub-set: beauty products by guerlain which are wishlisted on workdays
	%\item composition:intersect(low selling products && products from Vendor A && products which are lipsticks)
	%\item When it is added to a shopping bag its price may be decreased by applying a discount policy. These examples contain an initial event that is associated with a particular instance, which triggers a follow-up event. 
%\end{enumerate}




%\subsection{Explanation of features, just a summary}
%Selecting a set of objects based on their life-cycle events provide a new dimension in categorizing objects. Instance pointcuts reify such a set, by offering a declarative syntax for adding objects to it. The object to be added is exposed in the context of the matched event. An object might only be relevant during a particular phase of its life-cycle, which is marked by certain events. Then it is necessary to have a mechanism to remove objects from the set once the event that marks the end of that phase is encountered. Hence, it is possible to declare an optional removal expression in an instance pointcut.
%
%Instance pointcuts represent a set, and the objects in this set can be accessed through instance pointcuts. Also when there's a change in the set, i.e. an object is added or removed, it is  possible for the interested modules to access the change events. 
%
%Sets can have subsets or supersets. This is also true for the sets of objects created by instance pointcuts. An instance pointcut can be refined to obtain the subset of objects it originally selected or it can be a subset of another instance pointcut which broadens its scope. This refinement mechanism makes instance pointcuts reusable. Composition is also supported via the ability to perform set operations on instance pointcuts. It is possible to define intersection or union of two instance pointcuts, in order to define a new one. 
%
%The various features instance pointcuts offer also require some checking to be performed. For example when refining a pointcut to obtain a subset, we may end up with an empty set. This situation should be checked and necessary warning should be given. Since instance pointcuts dedicated to a type, the composition operations may fail if the composed pointcuts do not select co-variant types, which should also be checked.
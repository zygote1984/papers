% This is LLNCS.DOC the documentation file of
% the LaTeX2e class from Springer-Verlag
% for Lecture Notes in Computer Science, version 2.4
\documentclass{llncs}
\usepackage{llncsdoc}
%
\begin{document}

\title{SLE/AOSD Paper}
\maketitle
\section{Introduction}
Software evolution requires integration of new components with the old ones. In an
ideal component-based system, this integration should be seamless meaning 
 the legacy components remain untouched and the interface of the new component is fully compatible with the existing interfaces. 
Unfortunately such systems do not exist; as a result the integration is seldom
seamless.  Throughout the paper we will refer to the integration of two software components as a
\emph{binding}. We have defined three major challenges regarding binding. 

\begin{enumerate}
  	\item When components evolve, the links between them
	must be re-established. 
 	 \item When adding unforeseen functionality to a system, no explicit hooks
 	 exist for attaching the new component. In this case it may be necessary
  	to modify code to make the binding, which will unnecessarily expose the
  	developer to implementation details.
	\item There may be structures in the new and the old component, which
	conceptually overlap but provide different interfaces. 	The mapping between
	these structures should be defined, in order to avoid inconsistencies.
\end{enumerate}

Handling the first challenge requires a maintainable way of expressing binding. 

Solving the second challenge requires a means to choose certain points in an
application. \textbf{What would be a possible way to do this with OO and why is
it bad? Where is it limited? discuss middleware/containers here which have to provide
"explicit hooks"}
Solving the third challenge requires a means to express the communication and
the binding between components. \textbf{This is going to be explained after AOP
and how it offers solution to the first two challenges is mentioned.}

Aspect Oriented Programming (AOP) is designed to modularize crosscutting
concerns and with its 'weaving' mechanism, it is possible to change the behavior
(advice weaving) or the structure (AspectJ:inter-type declarations, CaesarJ:
family polymorphism, wrappers) of an implementation without altering the
implementation itself. These properties of AOP make it a desirable candidate for
attacking the binding problem.

Now let's go over the challenges again to understand how AOP can be of help.
AOP is effective for achieving loose-coupling. It
can capture information or inject behavior from a component without being
acknowledged. AOP also facilitates the OO way of loose-coupling, which done via
interfaces. It is possible to declare subtypes of an interface and provide the
subsequent implementation in an aspect.  AOP is efficient in localizing a concern. So when two components are bound using
AOP, the binding implementation will be in one place.This is an important
property for maintaining the modules providing loose-coupling. 

A pointcut is a program construct that selects join points and collects context at
those points \textbf{(AspectJ in Action book)}. Hence by using pointcuts we can
define entry points to a system to inject new behavior. Of course this approach
is limited with the expressiveness of the join-point model of the AO language.
But this is a potential AO solution for the second challenge.

The third challenge can also be addressed by AOP. However a solution in current
AO languages have drawbacks. \textbf{(I think here we should really discuss
JasCo)}. Let us look at a simple example and evaluate the possible solutions in
AspectJ and CaesarJ. 





\end{document}
